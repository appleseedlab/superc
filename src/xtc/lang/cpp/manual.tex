\documentclass{article}

\usepackage{amsmath}
\usepackage{amsthm}
\usepackage[colorlinks,urlcolor=black,citecolor=black,linkcolor=black]{hyperref}
\usepackage{graphicx}
\usepackage{algorithm}
\usepackage{algpseudocode}
\algrenewcommand{\algorithmiccomment}[1]{$\triangleright$ #1}
\usepackage{verbatim}
\usepackage[usenames,dvipsnames]{color}
\usepackage{colortbl}
\usepackage{rotating}
%\usepackage[normalem]{ulem}
\usepackage{listings}
\lstset{
  language=C,
  tabsize=8,
  numbers=left,
  captionpos=b,
  basicstyle=\scriptsize\ttfamily,
  numberstyle=\tiny\rmfamily,
  commentstyle=\scriptsize\ttfamily,
  breaklines=true,
  numbersep=5pt,
  columns=fullflexible
}

\newcommand{\SuperC}{{\textsf{Su\-perC}}}
\newcommand{\rats}{\textit{Rats!}}

\begin{document}

\title{The \SuperC{} Technical Manual}

\maketitle
\tableofcontents

%%%%%%%%%%%%%%%%%%%%%%%%%%%%%%%%%%%%%%%
%%%%%%%%%%%%%%%%%%%%%%%%%%%%%%%%%%%%%%%
%%%%%%%%%%%%%%%%%%%%%%%%%%%%%%%%%%%%%%%
\section{Usage}

%%%%%%%%%%%%%%%%%%%%%%%%%%%%%%%%%%%%%%%
\subsection{Environment}

SuperC requires three environment variables to run properly: (1)
\verb"JAVA_DEV_ROOT" should be set to the root of the \verb"xtc"
source tree.  (2) the \verb"PATH" should include
\verb"$JAVA_DEV_ROOT/xtc/lang/cpp/scripts/"; this directory contains
scripts to run and test \SuperC{}.  (3) \verb"JAVA_ARGS" should be set
to increase the heap and stack sizes.  The scripts directory contains
a script \verb"env.sh" that sets these and \verb"xtc"'s
\verb"CLASSPATH".

%%%%%%%%%%%%%%%%%%%%%%%%%%%%%%%%%%%%%%%
\subsection{Executables}

\begin{description}
\item[\texttt{java xtc.lang.cpp.SuperC}] - The
  configuration-preserving C parser, preprocessor, and lexer.

\item[\texttt{java xtc.lang.cpp.cdiff}] - C-token diff utility that
  uses SuperC's lexer and directiveParser.

\item[\texttt{java xtc.lang.cpp.FilenameService}] - provides a
  centralized distributor of filenames for distributed processing by
  \verb"superc_linux.sh".
\end{description}

%%%%%%%%%%%%%%%%%%%%%%%%%%%%%%%%%%%%%%%
\subsection{Core Scripts}

\begin{description}
\item[\texttt{data.sh}] - has the commented commands for all summary
  data used in the \SuperC{} paper.  It is not meant to be executed.

\item[\texttt{superc\_linux.sh}] - runs \SuperC{} on linux and
  provides other facilities related to testing \SuperC{} on linux,
  e.g., extracting header paths and comparing output to the Linux
  allyesconfig.

\item[\texttt{regression.sh}] - runs regression tests.  It is used by
  \verb"xtc"'s \verb"make check-cpp" target to regression test
  \SuperC{}.

\item[\texttt{typechef\_test.sh} and \texttt{typechef\_run\_file.sh}]
  - runs \SuperC{} and TypeChef on TypeChef's constrained kernel.

\item[\texttt{dynamic\_analysis.sh}] - summarizes the output of
  \texttt{-preprocessorStatistics} and \texttt{-languageStatistics}.


\end{description}

All of \SuperC{}'s scripts can be found in
\verb"$JAVA_DEV_ROOT/src/xtc/lang/cpp/scripts/".


%%%%%%%%%%%%%%%%%%%%%%%%%%%%%%%%%%%%%%%
%%%%%%%%%%%%%%%%%%%%%%%%%%%%%%%%%%%%%%%
%%%%%%%%%%%%%%%%%%%%%%%%%%%%%%%%%%%%%%%
\section{Instrumentation}

\SuperC{} and its scripts can report detailed statistics and analyses
of compilation units.  All instrumentation reports data as rows of
space-delimited columns.  The first column in each row is a keyword
identifying the type of data, e.g., \verb"define" for a macro
definition or \verb"max_subparsers" for the maximum subparsers used.
The format of each type of data is list in the following subsections.

%%%%%%%%%%%%%%%%%%%%%%%%%%%%%%%%%%%%%%%
\subsection{The \texttt{-preprocessorStatistics} Flag}

Preprocessor statistics are output when the \verb"-preprocessorStatistics"
flag is turned on.  Below is the format of the data output.

\begin{description}
\item[\texttt{define name fun|var location ndefs}] - will be output every
  time a macro definition is encountered.  \verb"ndefs" is the number of
  macros definitions in the conditional symbol table \emph{after}
  evaluating the definition.
\item[\texttt{undef name location ndefs}] - for every \verb"#undef". \verb"ndefs" is the
  number of definitions \emph{after} evaluating the undef.
\item[\texttt{object name location depth ndefs nused}] - for every
  object-like macro invocation.  \verb"depth" is the macro invocation
  nesting level.  \verb"ndefs" is the number of definitions, but \verb"nused" is
  the actual number of definitions expanded.  \verb"ndefs" is always
  greater than or equal to \verb"nused".  \verb"nused" is less when some macro
  definitions are trimmed.
\item[\texttt{function nargs name location depth ndefs nused}] - for every
  function-like macro invocation.  \verb"nargs" is the number of arguments
  passed to this function-like macro invocation.  \verb"depth", \verb"ndefs",
  and \verb"nused" are the same as for object-like macros.
\item[\texttt{hoist\_function name location depth nhoisted}] - every function-like
  macro is hoisted before invoking, even if their is only one
  resulting function.  \verb"nhoisted" shows how many function-like macro
  invocations resulted from hoisting.
\item[\texttt{paste token|conditional token|conditional location
    nhoisted nvalid}] - a token-pasting.  Each argument's type, token
  or conditional, is indicated.  \verb"nhoisted" is the number of
  token-paste operations resulting from the hoisting and \verb"nvalid"
  is the number of valid pastes out of the hoisted ones.
\item[\texttt{stringify token|conditional location nhoisted}] - a
  token-pasting.  The argument's type, token or conditional, is
  indicated.  \verb"nhoisted" is the number of stringification operations
  resulting from the hoisting.
\item[\texttt{include name resolved location (un)guarded system|user
    normal|next}] \texttt{single|computed depth} - a header include.
  Indicates given header name, header name resolved by the
  preprocessor, location of the directive, whether it was guarded,
  whether it was a system (\verb"<...>") or user header
  (\verb'"..."'), whether it was a gcc-specific \verb"#include_next"
  directive, whether it was from a computed header or not, and how
  many headers deep we are after entering this header.
\item[\texttt{computed location normal|next depth nhoisted}] - a computed
  include, indicating how many headers deep after performing the
  include, and the number of hoisted include directives.  Not all
  hoisted headers must be valid, so \verb"end_computed" will summarize how
  many actual headers were included.
\item[\texttt{end\_computed location nvalid}] - after a computed include has
  finished, this reports how many valid headers were hoisted.
\item[\texttt{conditional if|ifdef|ifndef|elif|else location
  boolean|nonboolean depth nhoisted}] - a start or next conditional
  directive. \verb"nonboolean" is emitted when the expression contains at
  least one non-boolean subexpression \verb"nhoisted" is the number of
  expressions resulting from multiply-defined macros in its
  expression.
\item[\texttt{endif location breadth}] - an endif, its location, and that
  number of branches in the conditional it is ending.
\item[\texttt{line|error\_directive|warning|pragma location}] - a line, error, warning,
  or pragma directive and its location.
\end{description}

In all cases, \verb"location" has the following format:

\begin{quote}
\verb"file:line:col[:macro]"
\end{quote}

where \verb"macro" is present only when the preprocessor usage occurs
inside of a macro expansion, e.g., nested macro expansion.

%%%%%%%%%%%%%%%%%%%%%%%%%%%%%%%%%%%%%%%
\subsection{The \texttt{-parserStatistics} Flag}

Parser statistics are output when the \verb"-parserStatistics" flag is
turned on.  Below is the format of the data output.

\begin{description}
\item[\texttt{iterations n}] - the number of iterations of the main parsing
  loop.
\item[\texttt{subparsers size iterations}] - this shows how many \verb"iterations"
  of the FMLR parsing loop had \verb"size" subparsers.
\item[\texttt{max\_subparsers size}] - this shows the maximum number of
  subparsers.
\item[\texttt{killswitch\_subparsers size}] - when -killswitch is used and
  the naive parser reaches the limit, this reports how many subparsers
  there were before killing the parser.
\item[\texttt{follow n}] - the number of times \verb"follow()" was called.
\item[\texttt{dag\_nodes n}] - the number of DAG nodes from the
  parser.  This represents the actual spaced used by the parser output
  in terms of nodes in memory.
\item[\texttt{dag\_nodes\_shared n}] - the number of DAG nodes that
  have more than one edge leading to it.
\item[\texttt{empty\_conditionals total with\_empty}]
\item[\texttt{lazy\_forks total with\_lazy}]
\end{description}

%%%%%%%%%%%%%%%%%%%%%%%%%%%%%%%%%%%%%%%
\subsection{The \texttt{-languageStatistics} Flag}

Language statistics are output when the \verb"-languageStatistics" flag is
turned on.  Below is the format of the data output.

\begin{description}
\item[\texttt{c\_construct name location}] - emitted every time a C
  statement or declaration is reduced.
\item[\texttt{typedef name location}] - emitted every time there a typedef
  name is bound.
\item[\texttt{typedef\_ambiguity name location}] - emitted every time a
  subparser encounters a name that is both a typedef name and a
  variable name in the same parsing context.
\item[*\texttt{conditional\_inside name location}] - emitted every
  time a C statement or declaration contains a conditional when it is
  reduced.  *Requires patching the parser (see
  Section~\ref{section:patching}).
\end{description}

%%%%%%%%%%%%%%%%%%%%%%%%%%%%%%%%%%%%%%%
\subsubsection{Patching the Parser to Report Embedded Conditionals}
\label{section:patching}

\SuperC{} can also report conditionals that are embedded in
declarations and statements when using \texttt{-languageStatistics}.
\SuperC{} is shipped with a patch that add extra fields to stack
frames and modifies the algorithm to track the conditionals and
generate output.  The patch can be applied in the \SuperC{} source
directory \verb"$JAVA_DEV_ROOT/src/xtc/lang/cpp/":

\begin{quote}
\begin{verbatim}
patch < instrument_embedded_conditionals.patch 
\end{verbatim}
\end{quote}

The patch can be reversed like this:

\begin{quote}
\begin{verbatim}
patch -R < instrument_embedded_conditionals.patch 
\end{verbatim}
\end{quote}


%%%%%%%%%%%%%%%%%%%%%%%%%%%%%%%%%%%%%%%
\subsection{Other \SuperC{} Flags}

\subsubsection{\texttt{-configurationVariables}}
\begin{description}
\item[\texttt{config\_var name}] - reports a configuration variable name
\end{description}

\subsubsection{\texttt{-headGuards}}
\begin{description}
\item[\texttt{header\_guard name}] - reports a header guard name
\end{description}

\subsubsection{\texttt{-size}}
\begin{description}
\item[\texttt{size bytes}] - reports the total compilation unit size in bytes
\end{description}

\subsubsection{\texttt{-time}}
\begin{description}
\item[\texttt{performance\_breakdown file parsing preprocessing
    lexing}] - reports the latencies in seconds of the file broken
  down by lexing alone, preprocessing plus lexing, and parsing plus
  preprocessing and lexing.
\end{description}

\subsubsection{\texttt{-killswitch}}
\begin{description}
\item[\texttt{killswitch subparsers}] - reports the number of
  subparsers reached if the killswitch is triggered
\end{description}


%%%%%%%%%%%%%%%%%%%%%%%%%%%%%%%%%%%%%%%
\subsection{The \texttt{superc\_linux.sh} Script}

\begin{description}
\item[\texttt{configs\_superc gcc-args}] - with \verb"-e", the gcc
  arguments for header include paths and the \verb"KBUILD_NAME"
  macros.
\item[\texttt{configs\_all gcc-args}] - with \verb"-k", the gcc
  arguments for header include paths and all configuration variables.
\item[\texttt{grammar\_succeeded file}] - with \verb"-P", for when the
  preprocessed version of the file was parsed successfully.
\item[\texttt{grammar\_failed file}] - with \verb"-P", for when the
  preprocessed version of the file was not parsed successfully.
\item[\texttt{performance file time}] - with \verb"-r", the running
  time in seconds of processing the file.
\item[\texttt{performance\_raw file before after}] - with \verb"-r",
  the start and end times in seconds of processing the file.  Output
  when the program \verb"bc" is not available to calculate the running
  time.
\item[\texttt{comparison\_succeeded file}] - with \verb"-c", for when
  the output of \SuperC{}, preprocessed, matches the tokens of the
  preprocessed original file.
\item[\texttt{comparison\_succeeded file}] - with \verb"-c", for when
  the output of \SuperC{}, preprocessed, does not match the tokens of
  the preprocessed original file.
\item[\texttt{comparison\_error file}] - with \verb"-c", when cdiff
  returned another error, meaning the test needs to be run again.
\item[\texttt{excluded file msg}] - When a file was skipped either
  because its configuration information was not found in the currently
  configured kernel or, with \verb"-w", its configuration was already
  written out to disk.
\end{description}

%%%%%%%%%%%%%%%%%%%%%%%%%%%%%%%%%%%%%%%
\subsection{The \texttt{typechef\_test.sh} Script}

\begin{description}
\item[\texttt{performance file seconds}] - reports the latency in
  seconds of the file, including JVM startup.
\end{description}

%%%%%%%%%%%%%%%%%%%%%%%%%%%%%%%%%%%%%%%
\subsection{The \texttt{dynamic\_analysis.sh} Script}

\begin{description}
\item[\texttt{summary\_definitions n in\_conditional nobject
    nfunction}] - the number of macro definitions, the number inside
  of conditionals, and the number that are object-like and
  function-like.
\item[\texttt{summary\_redefinitions n}] - the number of define directives that overrode the configuration of an existing definitions.
\item[\texttt{summary\_invocations n trimmed nobject nfunction}] - The
  number of macro invocations.  trimmed is the number of invocations
  where at least one definition was trimmed.
\item[\texttt{summary\_hoisted\_functions n}] - the number of
  function-like invocations that required conditionals to be hoisted
  around them.
\item[\texttt{summary\_nested\_invocations n}] - the number of macro
  invocations nested in other macro invocations.
\item[\texttt{summary\_paste n hoisted conditional\_arg}] - the number
  of token-asting operations.  hoisted is the number of paste
  operations with conditionals requiring hoisting.  conditional\_arg
  is the number of operations with a conditional argument (which
  should be the same as hoisted unless the conditional has only one
  branch and it's the same as the operation's presence condition).
\item[\texttt{summary\_stringify n hoisted conditional\_arg}] - the
  number of stringification operations.
\item[\texttt{summary\_include n computed hoisted valid\_hoisted}] -
  the number of file includes.  computed is the number of computed
  includes.  hoisted is the number of computed includes requiring
  hoisted.  valid\_hoisted is the number of computed includes with
  more than one valid resulting header file.
\item[\texttt{summary\_reinclude n}] - the number of includes that
  reinclude a previously-included header.
\item[\texttt{summary\_conditionals n nonboolean}] - the number of
  static conditionals (including all branches and endif).  nonboolean
  is the number of if and elif conditional directives with nonboolean
  subpexpressions in their expression.  maxdepth is the maximum
  nesting depth of conditionals, considering file inclusion.  hoisted
  is the number of if and elif conditionals directives whose
  expressions required hoisting of conditionals due to
  multiply-defined macros.
\item[\texttt{summary\_error\_directives n}] - the number of error
  directives.
\item[\texttt{summary\_c\_statement\_and\_declarations n
    with\_conditionals}] - the number of C statements and
  declarations, including external declarations.  with\_conditionals
  is the number of those constructs having embedded conditionals.
\item[\texttt{summary\_typedefs n}] - the number of typedef
  declarations.
\item[\texttt{summary\_typedef\_ambiguities n}] - the number of
  typedef names ambiguously-defined as an identifier.
\end{description}


%%%%%%%%%%%%%%%%%%%%%%%%%%%%%%%%%%%%%%%
%%%%%%%%%%%%%%%%%%%%%%%%%%%%%%%%%%%%%%%
%%%%%%%%%%%%%%%%%%%%%%%%%%%%%%%%%%%%%%%
\section{Implementing Languages}

\subsection{Overview}

The preprocessor and the parser are language-independent.  To use them
for a specific language, they need the following:

\begin{enumerate}
\item A token specification, e.g., \verb"c.l" for \SuperC{}.  This
  specifies the token name and regular expression for each token.  It
  also provides information to the preprocessor, specifically, which
  tokens are the comma, parentheses, hash mark, double hash mark, and
  ellipsis, and also which tokens can be used as macro names.  The
  header file \verb"token.h" provides preprocessor macros that make
  specifying tokens easy.
\item A grammar specification, e.g., \verb"c.y" for \SuperC{}.  This
  is a Bison grammar file annotated with AST-building instructions.
  \verb"xtc.lang.cpp.ActionsGenerator" will automatically generate an
  implementation of \verb"Actions.java" which provides the parser with
  AST-building information, e.g., \verb"CActionsBase.java".
  \verb"getValue()" must be implemented in a subclass of the generated
  class to tell the parser how to create semantic values, e.g.,
  \verb"CActions.java".
\item A token creator, e.g., \verb"CTokenCreator.java" for \SuperC{}.
  This provides the preprocessor with language-dependent methods to
  create new tokens and paste tokens.  \verb"TokenCreator.java"
  defines the interface.
\item (optional) An implementation of any semantic actions defined
  with the \verb"action" annotation in the grammar specification.
  \verb"xtc.lang.cpp.ActionsGenerator" will generate an abstract
  method for each action, e.g., in \verb"CActionsBase.java", that
  should be implemented in a subclass, e.g., \verb"CActions.java".
\item (optional) An implementation of the parsing context and its
  reclassification and merge methods.  The parsing context needs to
  implement the \verb"Actions.Context" interface, e.g.,
  \verb"CParsingContext.java"
\end{enumerate}

\subsection{Declarative AST Generation}

SuperC supports declarative AST generation.  The Bison grammar is
annotated with special comments next to a production's name that tell
SuperC's parser how to build the AST node.  The annotations must
appear right after the production's name between \verb'/**' and
\verb'**/' comment markers in a comma-delimited list.  Here are the
supported annotations and what they do:
\begin{description}
\item[list] flattens left-recursive lists into the same level.
\item[layout] is used for punctuation.  Its semantic value is null
  unless source is being preserved.
\item[void] always makes semantic value null.
\item[name(\textit{NodeName})] uses \textit{NodeName} for the AST
  node's name instead of the production's symol.
\item[passthrough] uses the child's semantic value for a production
  with only one child.  This flattens long chains of nested
  productions in, for example, C's expression grammar.
\item[action] makes the production a semantic action.  The production
  should be empty.
\item[complete] identifies a complete syntactic units.  Only these
  productions can go under conditional nodes.
\end{description}
Here is an example from the C grammar that defines external
declarations.
\begin{quote}
\begin{verbatim}
ExternalDeclaration:  /** passthrough, complete **/
        FunctionDefinitionExtension
        | DeclarationExtension
        | AssemblyDefinition
        | EmptyDefinition
        ;
\end{verbatim}
\end{quote}
Because of \textbf{passthrough}, the semantic value of this production
is taken from the child node, e.g.,
\verb'FunctionDefinitionExtension'.  Since this syntactic unit is
\verb'complete', it may appear directly under a conditional node.

\textbf{list}, \textbf{layout}, \textbf{void}, \textbf{passthrough},
and \textbf{action} are mutually-exclusive.  \textbf{complete} and
\textbf{name} can be added to any symbol.  \textbf{layout} is the only
annotation that can be used on tokens.

here are examples where each node declaration is used.

\subsection{Grammar Files and Semantic Actions}

SuperC's parser generator can extract java code from the bison grammar
file's prologue, epilogue, and inline semantic actions in and after
productions.  All semantic actions have access to a \texttt{subparser}
variable that is a reference to the current subparser reducing the
production, and therefore its stack and parsing context.  Semantic
actions can also set or modify the \texttt{value} variable.  The
contents of the value variable is the automatically-generated AST node
described in Declarative AST Generation above.

\end{document}
